\documentclass[12pt]{article}
\usepackage[utf8]{inputenc}
\usepackage[spanish]{babel}
\usepackage{graphicx}
\usepackage{eso-pic}
\usepackage{hyperref}
\usepackage{geometry}
\geometry{a4paper, margin=2.5cm}

\title{Tarea 1 \- Visualización de Datos\\\large Política en Chile}
\author{Grupo 1\\Sebastian Benavides\\ Bastian Gaete\\ Álvaro Sánchez\\}
\date{\today}

\begin{document}

\AddToShipoutPicture*{
    \put(30,720){\includegraphics[width=4cm]{imagenes/Logo_UTFSM.png}}
}
\maketitle

\section*{Criterios seleccionados}

Para esta tarea se analizaron dos criterios dentro del tema macro “Política”, enfocados en Chile:

\begin{enumerate}
    \item \textbf{Participación electoral por estado} en elecciones presidenciales desde 2000.
    \item \textbf{Composición del Congreso por partido político} desde el año 2000.
\end{enumerate}

Estos criterios fueron escogidos debido a que permiten observar tanto el nivel de involucramiento ciudadano como la evolución del poder legislativo, lo que da una visión amplia de la dinámica política de Estados Unidos.

\section*{Visualización 1: Participación electoral}

%\vspace{0.5cm}
%\includegraphics[width=\textwidth]{grafico1.png} % reemplazar por ruta del gráfico generado

\subsection*{Fuente de datos}
%United States Census Bureau: \url{https://www.census.gov/data/tables/time-series/demo/voting-and-registration.html}

\subsection*{Conclusiones}
Aquí puedes redactar tus conclusiones una vez generado el gráfico. Por ejemplo: \\
“Se observa que la participación electoral ha tenido una leve tendencia al alza desde 2000, con picos significativos en elecciones con alta polarización como 2008 y 2020.”

\section*{Visualización 2: Composición del Congreso}

%\vspace{0.5cm}
%\includegraphics[width=\textwidth]{grafico2.png} % reemplazar por ruta del gráfico generado

%\subsection*{Fuente de datos}
%Congress.gov: \url{https://www.congress.gov/resources/display/content/Statistics+and+Lists}

\subsection*{Conclusiones}
Ejemplo: \\
“Desde el año 2000, se observa una fuerte alternancia en el control de las cámaras, lo que refleja una polarización creciente. En los últimos años, el Congreso ha estado muy dividido, dificultando la gobernabilidad.”

\section*{Repositorio GitHub}
Puedes acceder al código, fuentes y visualizaciones en el siguiente enlace: \\
\url{https://github.com/tu-usuario/tarea1-politica-chile} \\

\end{document}
