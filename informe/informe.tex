\documentclass[12pt]{article}
\usepackage[utf8]{inputenc}
\usepackage[spanish]{babel}
\usepackage{graphicx}
\usepackage{eso-pic}
\usepackage{hyperref}
\usepackage{geometry}
\geometry{a4paper, margin=2.5cm}

\title{Tarea 1 \- Visualización de Datos\\\large Política en Chile}
\author{Grupo 1\\Sebastian Benavides\\ Bastian Gaete\\ Álvaro Sánchez\\}
\date{\today}

\begin{document}

\AddToShipoutPicture*{
    \put(30,720){\includegraphics[width=4cm]{imagenes/Logo_UTFSM.png}}
}
\maketitle

\section*{Introducción}

En esta tarea se seleccionó el tema macro “Política”, analizado desde distintas dimensiones por cada integrante del grupo. El objetivo es explorar diferentes aspectos de la política contemporánea mediante visualizaciones efectivas de datos, utilizando herramientas de Python y gráficos no convencionales.

% --- INTEGRANTE 1 --------------------------------------------------------------------

\section*{Dimensión: % [Escribir dimensión general aquí] 
% (Integrante: [Nombre 1])
}

\subsection*{Criterios seleccionados}
\begin{enumerate}
    \item \textbf{% [Criterio 1]
    }
    \item \textbf{% [Criterio 2]
    }
\end{enumerate}

\subsection*{Justificación}
% Escribir aquí por qué se eligieron los criterios anteriores. ¿Qué aportan al análisis político desde esta dimensión?

\subsection*{Visualización 1}
% \includegraphics[width=\textwidth]{grafico1.png} % Descomentar cuando tengas el gráfico

\textbf{Fuente de datos:} % [URL o fuente exacta]

\textbf{Conclusión:} \\
% [Conclusión basada en el gráfico 1]

\subsection*{Visualización 2}
% \includegraphics[width=\textwidth]{grafico2.png} % Descomentar cuando tengas el gráfico

\textbf{Fuente de datos:} % [URL o fuente exacta]

\textbf{Conclusión:} \\
% [Conclusión basada en el gráfico 2]

% --- INTEGRANTE 2 --------------------------------------------------------------------

\section*{Dimensión: % [Escribir dimensión general aquí] 
% (Integrante: [Nombre 2])
}

\subsection*{Criterios seleccionados}
\begin{enumerate}
    \item \textbf{% [Criterio 1]
    }
    \item \textbf{% [Criterio 2]
    }
\end{enumerate}

\subsection*{Justificación}
% Justificación de los criterios

\subsection*{Visualización 1}
% \includegraphics[width=\textwidth]{grafico3.png}

\textbf{Fuente de datos:} % [URL o fuente exacta]

\textbf{Conclusión:} \\
% [Conclusión basada en el gráfico 3]

\subsection*{Visualización 2}
% \includegraphics[width=\textwidth]{grafico4.png}

\textbf{Fuente de datos:} % [URL o fuente exacta]

\textbf{Conclusión:} \\
% [Conclusión basada en el gráfico 4]

% --- INTEGRANTE 3 --------------------------------------------------------------------

\section*{Dimensión: % [Escribir dimensión general aquí] 
% (Integrante: [Nombre 3])
}

\subsection*{Criterios seleccionados}
\begin{enumerate}
    \item \textbf{% [Criterio 1]
    }
    \item \textbf{% [Criterio 2]
    }
\end{enumerate}

\subsection*{Justificación}
% Justificación de los criterios

\subsection*{Visualización 1}
% \includegraphics[width=\textwidth]{grafico5.png}

\textbf{Fuente de datos:} % [URL o fuente exacta]

\textbf{Conclusión:} \\
% [Conclusión basada en el gráfico 5]

\subsection*{Visualización 2}
% \includegraphics[width=\textwidth]{grafico6.png}

\textbf{Fuente de datos:} % [URL o fuente exacta]

\textbf{Conclusión:} \\
% [Conclusión basada en el gráfico 6]

% --- GITHUB --------------------------------------------------------------------------

\section*{Repositorio GitHub}
El repositorio con el código y fuentes utilizadas por cada integrante se encuentra disponible en: \\
% \url{https://github.com/[usuario]/[repositorio]}

\end{document}